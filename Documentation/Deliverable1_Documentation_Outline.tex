\documentclass{article}
\usepackage{graphicx} 
\usepackage{etoolbox}
\usepackage{tabularx}
\usepackage[colorlinks=true, linkcolor=blue]{hyperref}
\usepackage{tabularx}
\usepackage{changepage} % for adjustwidth
\usepackage{array} % needed for defining new column types
\usepackage{changepage}

% Define a left-aligned X column
\newcolumntype{L}{>{\raggedright\arraybackslash}X}
\newcounter{subsubsubsection}[subsubsection]
\renewcommand{\thesubsubsubsection}{\alph{subsubsubsection})}
\renewcommand{\arraystretch}{1.2} % row height
\setlength{\tabcolsep}{6pt} 
\newcommand{\subsubsubsection}[1]{%
  \refstepcounter{subsubsubsection}%
  \vspace{0.5em}%
  \noindent{\normalsize\textbf{\textit{\thesubsubsubsection\ #1}}}\par
  \phantomsection
  \addcontentsline{toc}{subsubsection}{\hspace{2em}\thesubsubsubsection\ #1}%
}




\title{Deliverable 1: Documentation Outline}
\author{Group \#2}
\date{Fall 2025}

\begin{document}

\maketitle
\newpage

\setcounter{tocdepth}{4}
\tableofcontents
\newpage

\section{Group Members}
\begin{center}
\begin{tabular}{l l l}
Jeffrey Yueh & yuehj & 400495097 \\
Johnson Ji & jih21 & 400499564 \\
Kelby To & tok13 & 400507403 \\
Hongliang Qi & qih25 & 400493278 \\
\end{tabular}
\end{center}

\section{Part 1}

\subsection{Introduction}

\subsubsection{Purpose}
The purpose of a pacemaker is to regulate and restore a normal heart rhythm in patients with cardiac disorders such as arrhythmias, bradycardia, and heart failure. The pacemaker accomplishes this by delivering small electrical pulses to the atria and ventricles to make the heart beat at the correct speed and pattern. 

The system consists of two main components: the \textbf{Pacemaker} and the \textbf{Device Controller-Monitor (DCM)}. The Pacemaker handles sensing and pacing functions, while the DCM allows users to configure, monitor, and manage the settings of the Pacemaker.

\subsubsection{Goals}
The main goal of Deliverable 1 is to design and implement the foundational components of the Pacemaker and DCM. Specifically:
\begin{itemize}
    \item \textbf{Pacemaker:} Create stateflow models for AOO, VOO, AAI, and VVI modes with parameters specified in the Deliverable 1 document.
    \item \textbf{DCM:} Develop an interface that enables user registration and login, displays pacing modes, and allows input and storage of modifiable parameters.
    \item \textbf{Documentation:} Provide a detailed document outlining the design process, decisions, implementation, and testing procedures for the Pacemaker and DCM.
\end{itemize}

\subsubsection{Scope}
Deliverable 1 focuses on developing the initial components of the Pacemaker and DCM. This includes:
\begin{itemize}
    \item Creating stateflow models for the AOO, VOO, AAI, and VVI pacing modes.
    \item Building the DCM interface with registration, login, and parameter customization functionality.
    \item Documenting the design and implementation process.
\end{itemize}

This deliverable does not include advanced features such as wireless communication, complex arrhythmia detection algorithms, or additional pacing modes beyond those specified. Hardware testing is limited to simulation and software verification environments.

\subsection{Requirements}
\subsubsection{Modes}
\subsubsubsection{AOO Mode}
\begin{itemize}
    \item \textbf{Pacing:} Atrial only
    \item \textbf{Sensing:} Atrial
    \item \textbf{Response to Sensing:} Inhibited
    \item \textbf{Behaviour:} Paces the atrium only if no intrinsic atrial activity is sensed within the programmed interval.
\end{itemize}

\subsubsubsection{VOO Mode}
\begin{itemize}
    \item \textbf{Pacing:} Ventricular only
    \item \textbf{Sensing:} None
    \item \textbf{Response to Sensing:} None (asynchronous)
    \item \textbf{Behaviour:} Delivers ventricular pacing pulses at a fixed rate, regardless of intrinsic activity.
\end{itemize}

\subsubsubsection{AAI Mode}
\begin{itemize}
    \item \textbf{Pacing:} Atrial
    \item \textbf{Sensing:} Atrial
    \item \textbf{Response to Sensing:} Inhibited
    \item \textbf{Behaviour:} Paces the atrium only if no intrinsic atrial activity is sensed within the programmed interval
\end{itemize}

\subsubsubsection{VVI Mode}
\begin{itemize}
    \item \textbf{Pacing:} Ventricular
    \item \textbf{Sensing:} Ventricular
    \item \textbf{Response to Sensing:} Inhibited
    \item \textbf{Behaviour:} Paces the ventricle only if no intrinsic ventricular activity is sensed within the programmed interval
\end{itemize}


\subsection{Design}
\subsubsection{Pacemaker}
\subsubsection{DCM}
\subsubsubsection{Programmable Parameters}



In this section, you should expand on design decisions based on the requirements. Be specific about your system design and how components interact.

\begin{itemize}
    \item System architecture (major subsystems, hardware abstraction, pin mapping)
    \item Programmable parameters (rate limits, amplitudes, pulse widths, refractory periods, etc.)
    \item Hardware inputs and outputs (signals sensed, signals controlled)
    \item State machine design for each pacing mode (include diagrams or tables if applicable)
    \item Simulink diagram
    \item Screenshots of your DCM, explaining its software structure
\end{itemize}

Explicitly explain how each design decision maps directly to the stated requirements.

\section{Part 2}

\subsection{Requirements / Potential Changes}
\begin{tabularx}{\textwidth}{|l|L|L|}
\hline
\textbf{Module} & \textbf{Requirements} & \textbf{Potential Changes / Evolution} \\
\hline
HelpWindow & Display help documentation for parameters and pacing modes, with navigation and formatted text. & Adding new help topics (e.g., D2 modes), support for multimedia content, or online help updates. \\
\hline
ParamEnum & Store pacemaker modes and parameters, provide getter and setter interfaces with validation and stepping rules. & Adding new pacing modes (D2, AOOR, etc.), new parameters for advanced therapy modes. \\
\hline
ParameterManager  & Manage parameter operations: save, load, reset, and apply; GUI to edit parameters based on mode. & Improve validation rules, support bulk import/export, additional GUI elements for future parameters. \\
\hline
WelcomeWindow & Provide login and registration functionality; launch dashboard upon successful login. & Integration with secure authentication APIs, multi-user support, and potential cloud-based storage. \\
\hline
\end{tabularx}

\subsection{Design Decision / Potential Changes}
\begin{tabularx}{\textwidth}{|l|L|L|}
\hline
\textbf{Module} & \textbf{Design Decisions} & \textbf{Potential Changes / Revisions} \\
\hline
HelpWindow & Tkinter-based GUI; JSON files for content; text widget formatting. & Consider switching to web-based help, dynamic content loading, or modular UI frameworks. \\
\hline
ParamEnum & Python class with validated getters/setters; uses constants for mode definitions. & Might adopt database-driven parameters, use enums for clarity, or external config files for scalability. \\
\hline
ParameterManager  & GUI-bound parameter management; state tracking; method resolution via string names. & Refactor to MVC pattern, improve input validation, or integrate real-time device updates. \\
\hline
WelcomeWindow & Simple Tkinter GUI for login/registration; dashboard launch. & Upgrade authentication security, support OAuth, and redesign GUI for modern UX. \\
\hline
\end{tabularx}

\subsection{Module Description}

\paragraph{HelpWindow}
\begin{itemize}
    \item \textbf{Purpose:} Display help documentation for pacemaker modes and parameters.
    \item \textbf{Key functions/methods:} 
        \begin{itemize}
            \item Public: \texttt{update\_content}, \texttt{load\_help\_content} 
            \item Internal: \texttt{\_display\_param\_document}, \texttt{\_display\_mode\_document}, \texttt{\_display\_text\_content}
        \end{itemize}
    \item \textbf{Global/state variables:} \texttt{topics}, \texttt{current\_topic}, \texttt{content\_area}.
    \item \textbf{Interactions:} Reads JSON help files; updates GUI content dynamically.
\end{itemize}

\paragraph{ParamEnum}
\begin{itemize}
    \item \textbf{Purpose:} Store parameter values and enforce validation rules.
    \item \textbf{Key functions/methods:}
        \begin{itemize}
            \item Public: getters (\texttt{get\_*}) and setters (\texttt{set\_*}), \texttt{get\_default\_values}
            \item Internal: \texttt{\_is\_number}, \texttt{\_round\_to\_step}
        \end{itemize}
    \item \textbf{Global/state variables:} parameter values (e.g., \texttt{Lower\_Rate\_Limit}, \texttt{ARP}); \texttt{MODES} dictionary.
    \item \textbf{Interactions:} Used by \texttt{ParameterManager} and GUI modules.
\end{itemize}

\paragraph{ParameterManager}
\begin{itemize}
    \item \textbf{Purpose:} Enable saving, loading, resetting, applying, and editing parameters through a GUI.
    \item \textbf{Key functions/methods:}
        \begin{itemize}
            \item Public: \texttt{save\_params}, \texttt{load\_params}, \texttt{reset\_params}, \texttt{apply}, \texttt{save\_and\_round}
            \item Internal: \texttt{\_resolve\_method}, \texttt{\_getter\_candidates\_for\_key}, \texttt{\_setter\_candidates\_for\_key}, \texttt{\_mark\_unsaved}, \texttt{\_on\_close}
        \end{itemize}
    \item \textbf{Global/state variables:} \texttt{param\_entries}, \texttt{mode\_var}, \texttt{\_saved\_ok}, \texttt{param\_manager}.
    \item \textbf{Interactions:} Reads/writes JSON parameters; updates \texttt{ParamEnum} values; GUI reflects changes.
\end{itemize}

\paragraph{WelcomeWindow}
\begin{itemize}
    \item \textbf{Purpose:} Manage user authentication and launch the dashboard.
    \item \textbf{Key functions/methods:} \texttt{register}, \texttt{login}
    \item \textbf{Global/state variables:} \texttt{name\_entry}, \texttt{pass\_entry}, \texttt{root}.
    \item \textbf{Interactions:} Calls \texttt{register\_user} and \texttt{login\_user}; launches \texttt{DashboardWindow}.
\end{itemize}

\subsection{Testing}


\begin{adjustwidth}{-3cm}{-3cm} % shrink left and right margins by 1cm

%-------------------------
\textbf{HelpWindow Module}\\[2mm]
\begin{tabular}{|p{2.5cm}|p{3.5cm}|p{3cm}|p{5cm}|p{1.5cm}|}
\hline
\textbf{Test Case} & \textbf{Purpose} & \textbf{Input} & \textbf{Expected Output} & \textbf{Result} \\ \hline
Display help content & Ensure help topics load correctly & Open HelpWindow & Help text displayed for each topic, formatted correctly & Pass \\ \hline
Switch topic & Verify topic switching & Click different topic button & Content area updates with selected topic & Pass \\ \hline
Missing JSON file & Check error handling & Remove Param\_Help.json & Display error message in content area & Pass \\ \hline
\end{tabular}

\vspace{5pt}
%-------------------------
\textbf{ParamEnum Module}\\[2mm]
\begin{tabular}{|p{3cm}|p{3.5cm}|p{4.5cm}|p{5cm}|p{1.5cm}|}
\hline
\textbf{Test Case} & \textbf{Purpose} & \textbf{Input} & \textbf{Expected Output} & \textbf{Result} \\ \hline
Set Lower Rate Limit & Validate stepping rules & \raggedright \texttt{set\_Lower\_Rate\_Limit(47)} & Value rounded to nearest step (45 or 50 depending on rule) & Pass \\ \hline
Set Upper Rate Limit & Check upper bound & \raggedright \texttt{set\_Upper\_Rate\_Limit(180)} & Error raised: out of range & Pass \\ \hline
Get default values & Ensure getters return defaults & \raggedright \texttt{get\_default\_values()} & Dictionary with correct default values & Pass \\ \hline
\end{tabular}


\vspace{5pt}
%-------------------------
\textbf{ParameterManager Module}\\[2mm]
\begin{tabular}{|p{3cm}|p{3.5cm}|p{3cm}|p{5cm}|p{1.5cm}|}
\hline
\textbf{Test Case} & \textbf{Purpose} & \textbf{Input} & \textbf{Expected Output} & \textbf{Result} \\ \hline
Save parameters & Verify JSON saving & Edit parameters, click Save & parameters.json created/updated & Pass \\ \hline
Load parameters & Verify JSON loading & Click Load & Entries updated with stored values & Pass \\ \hline
Apply parameters & Check apply function & Modify parameters, click Apply & Mode applied, entries validated & Pass \\ \hline
Reset parameters & Check reset functionality & Click Reset & All parameters revert to defaults & Pass \\ \hline
Validation & Input invalid values & Enter 200 for Upper Rate Limit & Error message displayed & Pass \\ \hline
\end{tabular}

\vspace{5pt}
%-------------------------
\textbf{WelcomeWindow Module}\\[2mm]
\begin{tabular}{|p{3cm}|p{3.5cm}|p{3cm}|p{5cm}|p{1.5cm}|}
\hline
\textbf{Test Case} & \textbf{Purpose} & \textbf{Input} & \textbf{Expected Output} & \textbf{Result} \\ \hline
User registration & Validate registration & Enter name/password, click Register & Success message or error if user exists & Pass \\ \hline
User login success & Test login workflow & Enter valid credentials, click Login & Launch DashboardWindow & Pass \\ \hline
User login failure & Test invalid login & Enter wrong credentials & Display error message & Pass \\ \hline
\end{tabular}


\vspace{5pt}
%-------------------------
\textbf{WelcomeWindow Module}\\[2mm]
\begin{tabular}{|p{3cm}|p{3.5cm}|p{3cm}|p{5cm}|p{1.5cm}|}
\hline
\textbf{Test Case} & \textbf{Purpose} & \textbf{Input} & \textbf{Expected Output} & \textbf{Result} \\ \hline
User registration & Validate registration & Enter name/password, click Register & Success message or error if user exists & Pass \\ \hline
User login success & Test login workflow & Enter valid credentials, click Login & Launch DashboardWindow & Pass \\ \hline
User login failure & Test invalid login & Enter wrong credentials & Display error message & Pass \\ \hline
\end{tabular}

\end{adjustwidth}

\subsection{GenAI Usage}
Provide a summary of any usage of GenAI tools in developing the model, DCM, or documentation. If no GenAI tools were used, state that explicitly.

\section{General Notes}
\begin{itemize}
    \item This outline is based on the Deliverable 1 handout; ensure all required sections are included.
    \item Include screenshots of Simulink diagrams and the DCM interface where applicable.
    \item Ensure requirements are traceable to design and test cases.
    \item Keep content concise and clear.
    \item You may add other sections or modify this structure as needed, but these are the main expected components.
\end{itemize}

\newpage
\section{Figures and Tables}
\begin{figure}[h]
	\centering
	\includegraphics[width=\textwidth]{Screenshot 2025-10-18 142545.png}
	\caption{Login Screen}
	\label{fig:login}
\end{figure}
\begin{figure}[h]
	\centering
	\includegraphics[width=\textwidth]{Screenshot 2025-10-18 143619.png}
	\caption{Home Screen}
	\label{fig:home}
\end{figure}
\end{document}
